%\begin{document}
\section{Thesis organisation}
This thesis is organised as follows:
In Chapter~\ref{ch:state-art}, the state of the art of the additive
manufacturing technology, \gls{lbam}, and \gls{mmam} is presented, with special
focus on the last, namely on \gls{fgm} structures. Lastly, a brief overview of
the available methodologies in these fields are presented.

In Chapter~\ref{ch:theor-found} the theoretical foundations are presented, namely the project development methodologies and associated tools,
and the \gls{sls}/\gls{slm} process in detail.

In Chapter~\ref{ch:prob-challenge}, the multi-material design and production
problem and its challenges using \gls{sls}/\gls{slm} technology are
presented. The methodology devised for multi-material production through the
\gls{lbam} technology is presented to tackle the high complexity of the process
and the lack of a supporting methodology, taking into account the key agents of
the process and leveraging the process information.

In Chapter~\ref{ch:development} is presented all the development phase of the
project. A specific workflow was instantiated from
the methodology, attending to the specific requirements and constraints of the
project. Based on this workflow, a toolchain was assembled, designing the
required software components. Finally, based on the requirements and constraints
of the process itself, the mechanical and electronic infrastructures were
designed, and on top of the last, the control software was designed.

In Chapter~\ref{ch:application}, the workflow and equipment were put to the test
to verify their suitability to the process and their performance for
multi-material component production. Additionally, production manufacturing tests
were also performed. Tests were used to validate the workflow and equipment,
pointing out also straightforward ways to adapt and implement custom paths for
the multi-material \gls{lbam} process.

The Chapter~\ref{ch:conclusion} gives a summary of this thesis as well as
prospect for future work.

Lastly, the appendices (see Section~\ref{ch:Append}) contain detailed information...
%%% Local Variables:
%%% mode: latex
%%% TeX-master: "../../../dissertation"
%%% End:
